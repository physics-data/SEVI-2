\documentclass{article}
\usepackage[utf8x]{inputenc}
\usepackage{ctex,titletoc,array,graphicx,float,subfigure,booktabs,multirow,colortbl,geometry,pgffor,indentfirst,physics,tikz,extarrows,tablefootnote}
\usepackage{cite}
\usepackage[]{hyperref} 
\usepackage{amsmath}
\usepackage{amssymb}
\usepackage{amsfonts}
\usepackage{mathrsfs}
% \usepackage[marginal]{footmisc}%脚注
% \usepackage[table]{xcolor}
\usepackage[final]{pdfpages}%插入pdf文档
\usepackage[justification=centering]{caption}%图表标题居中
\usepackage{setspace}%调整行距
\usepackage{chngpage}%调整页边距
% \usepackage{animate}%动图
% \usepackage{media9}%动图按钮
\usepackage{listings}
\usepackage{minted}
\usemintedstyle{emacs}
\usepackage{pmboxdraw}
\usepackage[bottom]{footmisc}
\definecolor{highlight1}{cmyk}{0.2,0.2,0.2,0}
\usepackage{siunitx}
\usepackage[version=3]{mhchem}
\newcommand{\pp }{\partial }%求偏导的简写:\pp
\newcommand{\ii }{\text{i} }%虚数单位简写:\ii
\newcommand{\deriv}[2]{\frac{\mathrm{d}\,#1}{\mathrm{d}\,#2}}
\definecolor{pureblue}{rgb}{0,0,1}
\hypersetup{citecolor=pureblue,CJKbookmarks=true,colorlinks=true,linkcolor=blue}%文献引用颜色
\bibliographystyle{unsrt}
\newcommand{\fra}[2]{\frac{\displaystyle #1}{\displaystyle #2}}
% \renewcommand{\cite }[1]{\textsuperscript{\cite{#1}}}
\titlecontents{section}[0mm]
                       {\vspace{.2\baselineskip}\bfseries}
                       {\thecontentslabel~\hspace{.5em}}
                       {}
                       {\dotfill\contentspage[{\makebox[0pt][r]{\thecontentspage}}]}
                       [\vspace{.1\baselineskip}]

\title{原子分子第二阶段大作业}
\author{实验物理的大数据方法}
\date{\today}

\begin{document}

\maketitle

\section{总体要求} % (fold)
\label{sec:总体要求}
我们会使用\href{https://github.com/physics-data/tpl_SEVI}{第一阶段}的方法模拟实验数据,但模拟的设置会略有差异。电子的速度场将更为复杂。首先,由于存在多种电离能,最终发射的电子球速度有多个值,意味着成像时有多个电子球壳。对于第 j 层级的电子,$r \sim N(R_j, \sigma_{R_j}^2)$,而 $\displaystyle\theta \sim I_j(\theta) = \frac{1}{2} + \sum_{i=0}^{\infty}\beta_{2i} P_{2i}(\cos\theta)$\footnote{$P_l$ 是 $l$ 阶勒让德函数。注意,勒让德函数正交归一条件为 $\displaystyle\langle P_l | P_n \rangle = \int_{-1}^{1} P_l(x)P_n(x)\dd x = \delta_{ln} \frac{2}{2l+1}$},你的任务是尽可能精准重建出各向异性参数 $\beta_{ij}$;$R_j$ 与 $\sigma_{R_j}$ 也是未知的,精准重建这些值也是重要的(不然 $\beta_{ij}$ 也会不准),不过不纳入最终评分指标。

为了简单起见,设置 $j \in \{1,2\}$,即最终有两个电子球。你将获得与\href{https://github.com/physics-data/tpl_SEVI}{第一阶段}输出一样的输入数据,即 50000 张图像,然后辨别电子的位置,并反推出电子动量分布。

因为本项大作业也不是助教的专业领域,所以出题时可能存在诸多不明晰的问题。有任何疑问,请在\href{https://github.com/physics-data/faq/discussions}{讨论区}发帖探讨。

% section 总体要求 (end)

\section{数据格式} % (fold)
\label{sec:数据格式}

所有数据均采用 HDF5 格式。

\subsection{测试数据与提交数据格式} % (fold)
\label{sub:测试数据与提交数据格式}
最终评分用的测试数据集只包含图像,包含一个有 50000 组数据的表 “FinalImage”:

\begin{table}[H]
\caption{FinalImage 表格}
    \label{tab:FinalImage}
    \centering
    {
        \begin{tabular}[c]{l|l|l}
            \hline
            \multicolumn{1}{c|}{\textbf{名称}} & 
            \multicolumn{1}{c|}{\textbf{说明}} & 
            \multicolumn{1}{c}{\textbf{类型}} \\
            \hline
            ImageId & 图片编号 & '<i4' \\
            Image & 单次实验所成图像 & 'u1', (1024,1024)\tablefootnote{'u1' 是 unsigned int8} \\
            \hline
        \end{tabular}
    }
\end{table}

与第一阶段不同,第二阶段之所以使用 1024$\times$1024 大小的图像,主要是因为 2048$\times$2048 分辨率较高,电子识别较为容易;1024$\times$1024 更加贴合实际情况,且数据占用空间较小。此外,为了进一步减小数据集的大小(以便下载),图像没有加高斯噪声。

处理之后,你需要重建电子动量分布的参数 $R_1$,$R_2$,$\{\beta_{i1}\}$ 与 $\{\beta_{i2}\}$。$\{\beta_{ij}\}$ 的个数上不封顶,但模拟中是有截断的。你也需要对 $\{\beta_{ij}\}$ 给出自己的截断,即当 $i>N$ 时 $\beta_{ij} = 0$。此时你只需要输出数据中存储前 $N$ 位 $\beta_{ij}$。

你需要在输出结果中包含一张表格 “Answer”,它有两行数据,一行对应电离能 1,另一行对应电离能 2:

\begin{table}[H]
    \caption{Answer 表格}
        \label{tab:Answer}
        \centering
        {
            \begin{tabular}[c]{l|l|l}
                \hline
                \multicolumn{1}{c|}{\textbf{名称}} & 
                \multicolumn{1}{c|}{\textbf{说明}} & 
                \multicolumn{1}{c}{\textbf{类型}} \\
                \hline
                SphereId & 球壳的编号,默认半径小的球壳在前 & '<u1' \\
                $R$ & 球壳半径 & '<f8' \\
                beta& 电子动量 $\theta$ 分布各向异性参数 & '<f8', (N) \\
                \hline
            \end{tabular}
        }
\end{table}

% subsection 测试数据与提交数据格式 (end)

\subsection{训练数据格式} % (fold)
\label{sub:训练数据格式}

我们会提供了两类训练数据:第一类是完整进行\href{https://github.com/physics-data/tpl_SEVI}{第一阶段}模拟的,包含电子动量初始信息、电子打击在 MCP 上的位置、荧光屏上的图像。由于图像占用空间较大,这类数据的数据量有限,因此设置了第二类数据集。第二类数据集会给出不同的 $R_j$、$\sigma_{R_j}$ 以及 $\beta_{ij}$ 参数,及其生成的电子动量 $(r,\theta)$、电子打在 MCP 的位置 $(x,z)$。

第一类训练数据包含 个表格。“Truth” 表格记录着电子动量分布的表达式,格式与表 \ref{tab:Answer} 相似:

\begin{table}[H]
    \caption{Truth 表格}
        \label{tab:Truth}
        \centering
        {
            \begin{tabular}[c]{l|l|l}
                \hline
                \multicolumn{1}{c|}{\textbf{名称}} & 
                \multicolumn{1}{c|}{\textbf{说明}} & 
                \multicolumn{1}{c}{\textbf{类型}} \\
                \hline
                SphereId & 球壳的编号 & '<u1' \\
                $R$ & 球壳半径 & '<f8' \\
                $\sigma_R$ & 球壳半径展宽 & '<f8' \\
                beta& 电子动量 $\theta$ 分布各向异性参数 & '<f8', (N)\tablefootnote{N 取决于具体数据} \\
                \hline
            \end{tabular}
        }
\end{table}

“FinalImage” 包含最终生成的图像,与测试数据中的表 \ref{tab:FinalImage} 一致。此外还有“GeneratedElectrons” 与 “DetectedElectrons”:

\begin{table}[H]
    \caption{GeneratedElectrons 表格}
        \label{tab:GeneratedElectrons}
        \centering
        {
            \begin{tabular}[c]{l|l|l}
                \hline
                \multicolumn{1}{c|}{\textbf{名称}} & 
                \multicolumn{1}{c|}{\textbf{说明}} & 
                \multicolumn{1}{c}{\textbf{类型}} \\
                \hline
                SphereId & 球壳的编号,与 Truth 表中的球壳编号对应 & '<u1' \\
                ElectronId & 电子编号 & '<i4' \\
                $r$ & 电子动量大小(与半径等价) & '<f8' \\
                $\cos\theta$ & 电子动量极角余弦 & '<f8' \\
                \hline
            \end{tabular}
        }
\end{table}

\begin{table}[H]
    \caption{DetectedElectrons 表格}
        \label{tab:DetectedElectrons}
        \centering
        {
            \begin{tabular}[c]{l|l|l}
                \hline
                \multicolumn{1}{c|}{\textbf{名称}} & 
                \multicolumn{1}{c|}{\textbf{说明}} & 
                \multicolumn{1}{c}{\textbf{类型}} \\
                \hline
                ImageId & 图像编号,与 FinalImage 表中的图像编号对应 & '<i4' \\
                ElectronId & 电子编号,与 GeneratedElectrons 中的电子编号对应 & '<i4' \\
                $x$ & 被 MCP 倍增的电子在 MCP 平面上的横坐标 & '<f8' \\
                $z$ & 被 MCP 倍增的电子在 MCP 平面上的纵坐标 & '<f8' \\
                \hline
            \end{tabular}
        }
\end{table}

而第二类训练数据不包含图像,但会包含多种电子动量分布参数。包含三个表格:

\begin{table}[H]
    \caption{DistributionParameters 表格}
        \label{tab:DistributionParameters}
        \centering
        {
            \begin{tabular}[c]{l|l|l}
                \hline
                \multicolumn{1}{c|}{\textbf{名称}} & 
                \multicolumn{1}{c|}{\textbf{说明}} & 
                \multicolumn{1}{c}{\textbf{类型}} \\
                \hline
                DistributionId & 动量分布编号(其实等价于球壳编号) & '<i4' \\
                $R$ & 球壳半径 & '<f8' \\
                $\sigma_R$ & 球壳半径展宽 & '<f8' \\
                beta& 电子动量 $\theta$ 分布参数 & '<f8', (N)\\
                \hline
            \end{tabular}
        }
\end{table}

\begin{table}[H]
    \caption{GeneratedElectrons 表格}
        \label{tab:GeneratedElectrons2}
        \centering
        {
            \begin{tabular}[c]{l|l|l}
                \hline
                \multicolumn{1}{c|}{\textbf{名称}} & 
                \multicolumn{1}{c|}{\textbf{说明}} & 
                \multicolumn{1}{c}{\textbf{类型}} \\
                \hline
                DistributionId & 动量分布编号 & '<i4' \\
                ElectronId & 电子编号 & '<i4' \\
                $r$ & 电子动量大小(与半径等价) & '<f8' \\
                $\cos\theta$ & 电子动量极角余弦 & '<f8' \\
                \hline
            \end{tabular}
        }
\end{table}

\begin{table}[H]
    \caption{DetectedElectrons 表格}
        \label{tab:DetectedElectrons2}
        \centering
        {
            \begin{tabular}[c]{l|l|l}
                \hline
                \multicolumn{1}{c|}{\textbf{名称}} & 
                \multicolumn{1}{c|}{\textbf{说明}} & 
                \multicolumn{1}{c}{\textbf{类型}} \\
                \hline
                ElectronId & 电子编号,与 GeneratedElectrons 中的电子编号对应 & '<i4' \\
                $x$ & 电子在 MCP 平面上的横坐标 & '<f8' \\
                $z$ & 电子在 MCP 平面上的纵坐标 & '<f8' \\
                \hline
            \end{tabular}
        }
\end{table}


% subsection 训练数据格式 (end)

% section 数据格式 (end)

\section{黑盒评分函数} % (fold)
\label{sec:黑盒评分函数}
记测试数据实际使用的分布各向异性参数为 $\beta_{ij}, j\in \{1,2\}$;提交的答案重建出的参数为 $\hat{\beta}_{ij}$,则最终得分 Distance = $\displaystyle\sum_{j=1}^{2}\sqrt{\sum_{i=0}^{\infty} \frac{2}{4i+1} \left( \hat{\beta}_{ij} - \beta_{ij} \right)^2}$。分数越低,排名越高。
% section 黑盒评分函数 (end)

\section{白盒分数} % (fold)
\label{sec:白盒分数}
与其它大作业一样,白盒部分占剩余的 20\%。分为 Git 使用(5\%)、代码风格(5\%)与实验报告(10\%)。在实验报告中,需要包含以下内容:
\begin{itemize}
    \item 处理算法的核心思想(包括必要的数学推导、文献引用等)
    \item 代码整体结构与思路
    \item 遇到的问题及解决方法
    \item 代码的运行方式(特别是使用了额外的软件包或者需要较复杂的配置时)
\end{itemize}
    
% section 白盒分数 (end)
\end{document}